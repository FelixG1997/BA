\documentclass{scrartcl}


% input encoding
\usepackage[utf8]{inputenc}

% new german spelling
\usepackage[ngerman]{babel}

% choose font
\usepackage[T1]{fontenc}
\usepackage{lmodern}

% KOMA-Script options
\KOMAoptions{%
  parskip=full,%
  fontsize=12pt,%
  DIV=calc}

% page options
\usepackage{geometry}
\geometry{a4paper, portrait, left=2.5cm, right=2cm, top=2cm, bottom=2cm, includefoot}

% color and images
\usepackage{xcolor}
\usepackage{graphicx}

% draw with tikz
\usepackage{tikz}
\usetikzlibrary{positioning,automata}

% quotes
\usepackage[german=guillemets]{csquotes}

% better enumerations
\usepackage{enumitem}

% title spacing fix
%\usepackage{titlesec}
%\titlespacing*{\subsection}{0pt}{2\baselineskip}{-0.5\baselineskip}
\RedeclareSectionCommand[
  beforeskip=-2\baselineskip,
  afterskip=.5\baselineskip]{section}
\RedeclareSectionCommand[
  beforeskip=-1.5\baselineskip,
  afterskip=.1\baselineskip]{subsection}
\RedeclareSectionCommand[
  beforeskip=-1\baselineskip,
  afterskip=.05\baselineskip]{subsubsection}
\RedeclareSectionCommand[
  beforeskip=.5\baselineskip,
  afterskip=-1em]{paragraph}
\RedeclareSectionCommand[
  beforeskip=-.5\baselineskip,
  afterskip=-1em]{subparagraph}

% math
\usepackage{amsmath}
\usepackage{amssymb}
\usepackage{amsthm}

% custom theorems
\usepackage{thmtools}
\declaretheorem[style=definition]{aufgabe}

% set special behaviour for hyperlinks in pdfs
\usepackage[breaklinks=true]{hyperref}

% display program code
\usepackage{listings}

\lstset{ %
  backgroundcolor=\color{white!98!black},  
  basicstyle=\ttfamily\footnotesize,
  breakatwhitespace=true,
  breaklines=true,
  commentstyle=\color{green!40!black},
  frame=single,
  keywordstyle=\color{blue!70!black},
  language=ML,
  numbers=left,
  numbersep=8pt,
  numberstyle=\scriptsize\color{black!80}, 
  rulecolor=\color{black!70},
  stepnumber=1,
  stringstyle=\color{purple!80!black},
  tabsize=2,
  title=\lstname,
  language=ML,
  belowskip=-0.8\baselineskip,
  aboveskip=-0.8 \baselineskip,
}

% Pseudocode
\usepackage{algorithm}
\usepackage{algpseudocode}

% positioning
\usepackage{float}

% document information
\title{Moving-Target TSP in two-orthogonal-axes}
\subtitle{Pseudocode-BF}
\author{Felix Greuling (666020)}
\date{21. Oktober 2019}

\begin{document}
\maketitle

% if, state, for, endif, endfor, else
% \Procedure{CalculateFunSums}{$T, w, visited, V$}
\begin{minipage}{1\linewidth}
\begin{algorithm}[H]
\begin{algorithmic}
\floatname{algorithm}{Algorithmus}
\caption{Brute-Force-Algorithmus für zwei-orthogonale Achsen beim bewegende Ziele in TSP}
\label{alg:BF}
\State \textbf{Input:} Ziele $Z$, Ursprung $z_{ursprung}$, Verfolgergeschwindigkeit $v_{max}$
\State \textbf{Output:} Ziele $Z$ in der Tour-Reihenfolge, inklusive Retour zum Ursprung\\

\State $B\rightarrow$ Permutationsbaum, bei $n$ Zielen muss es $n$ Kinderknoten geben.
\State $\tau_{min}\leftarrow \infty$
\State Sei $aktuelleZiele$ das Array mit den aktuellen Indizes von $Z$, initialisiert mit einsen
\State Sei $k$ die aktuelle Tiefe im Baum
\State Sei $t$ das Zeit-Array welches für die aktuelle Ziel-Reihenfolge die Abfangzeit angibt
\State $k\leftarrow 1$

\While {$aktuelleZiele[0]\leq n$ und $aktuelleZiele[n]\leq n$}
\State $current\leftarrow Z[aktuelleZiele[k]]$
\State $prev\leftarrow Z[aktuelleZiele[k-1]]$
\State $t[current] \leftarrow t[prev] + \pi[prev\rightarrow current]$
\If {$t[current]\geq \tau_{min}$ oder mindestens ein Ziel der verbliebenen Indizes wird zwischen\\~~~~~~~ $current$ und $prev$ abgefangen}
\If {$aktuelleZiele[k]=n$}
\State Dekrementiere $k$
\State Inkrementiere $aktuelleZiele[k]$
\Else
\State Inkrementiere $aktuelleZiele[k]$ 
\EndIf
\Else
\If {$k<n$}
\State Inkrementiere $k$
\Else
\State $t[current]\leftarrow t[current] + \pi[current\rightarrow ursprung]$ 
\If {$t[current] < \tau_{min}$}
\State $\tau_{min}\leftarrow t[current]$
\State Dekrementiere $k$
\State Inkrementiere $aktuelleZiele[k]$
\EndIf 
\EndIf
\EndIf
\EndWhile
\end{algorithmic}
\end{algorithm}
\end{minipage}
\end{document}