\documentclass{scrartcl}


% input encoding
\usepackage[utf8]{inputenc}

% new german spelling
\usepackage[ngerman]{babel}

% choose font
\usepackage[T1]{fontenc}
\usepackage{lmodern}

% KOMA-Script options
\KOMAoptions{%
  parskip=full,%
  fontsize=12pt,%
  DIV=calc}

% page options
\usepackage{geometry}
\geometry{a4paper, portrait, left=2.5cm, right=2cm, top=2cm, bottom=2cm, includefoot}

% color and images
\usepackage{xcolor}
\usepackage{graphicx}

% draw with tikz
\usepackage{tikz}
\usetikzlibrary{positioning,automata}

% quotes
\usepackage[german=guillemets]{csquotes}

% better enumerations
\usepackage{enumitem}

% title spacing fix
%\usepackage{titlesec}
%\titlespacing*{\subsection}{0pt}{2\baselineskip}{-0.5\baselineskip}
\RedeclareSectionCommand[
  beforeskip=-2\baselineskip,
  afterskip=.5\baselineskip]{section}
\RedeclareSectionCommand[
  beforeskip=-1.5\baselineskip,
  afterskip=.1\baselineskip]{subsection}
\RedeclareSectionCommand[
  beforeskip=-1\baselineskip,
  afterskip=.05\baselineskip]{subsubsection}
\RedeclareSectionCommand[
  beforeskip=.5\baselineskip,
  afterskip=-1em]{paragraph}
\RedeclareSectionCommand[
  beforeskip=-.5\baselineskip,
  afterskip=-1em]{subparagraph}

% math
\usepackage{amsmath}
\usepackage{amssymb}
\usepackage{amsthm}

% custom theorems
\usepackage{thmtools}
\declaretheorem[style=definition]{aufgabe}

% set special behaviour for hyperlinks in pdfs
\usepackage[breaklinks=true]{hyperref}

% display program code
\usepackage{listings}

\lstset{ %
  backgroundcolor=\color{white!98!black},  
  basicstyle=\ttfamily\footnotesize,
  breakatwhitespace=true,
  breaklines=true,
  commentstyle=\color{green!40!black},
  frame=single,
  keywordstyle=\color{blue!70!black},
  language=ML,
  numbers=left,
  numbersep=8pt,
  numberstyle=\scriptsize\color{black!80}, 
  rulecolor=\color{black!70},
  stepnumber=1,
  stringstyle=\color{purple!80!black},
  tabsize=2,
  title=\lstname,
  language=ML,
  belowskip=-0.8\baselineskip,
  aboveskip=-0.8 \baselineskip,
}

% Pseudocode
\usepackage{algorithm}
\usepackage{algpseudocode}

% positioning
\usepackage{float}

% document information
\title{Moving-Target TSP in two-orthogonal-axes}
\subtitle{Pseudocode}
\author{Felix Greuling (666020)}
\date{21. Oktober 2019}

\begin{document}
\maketitle

% if, state, for, endif, endfor, else
% \Procedure{CalculateFunSums}{$T, w, visited, V$}

\begin{minipage}{1\linewidth}
\begin{algorithm}[H]
\caption{Exact Algorithm for two-orthogonal-axes Moving-Target TSP}
\begin{algorithmic}
\State \textbf{Input:} The initial positions and velocities of n targets, and the maximum pursuer speed
\State \textbf{Output:} A time-optimal tour intercepting all targets, and returning back to the origin \\
\State \textbf{Preprocessing}
\State \emph{Use Preprocessing to eliminate targets to optimize the tour}
\State Partition the list of targets into the targets on the left side, the right side, the top side and the bottom side of the origin\\
\If {all remaining targets in $PrioQ$ are located on one side of the intersection,\\~~~ once the pursuer reached the intersection from \emph{current}}
\State Calculate the time required to intercept all remaining targets (and return 
\State to the origin), sort them in order of nondecreasing interception times and 
\State add all to $OUTPUT$
\State Go to the \emph{Postprocessing} step
\EndIf 
\\\\
Sort the targets on the left into list \emph{Left} in order of nonincreasing speeds \\
Sort the targets on the right into list \emph{Right} in order of nonincreasing speeds \\
Sort the targets on the top into list \emph{Top} in order of nonincreasing speeds \\
Sort the targets on the bottom into list \emph{Bottom} in order of nonincreasing speeds \\

\State Delete targets in \emph{Left, Right, Top and Bottom} which are closer to the origin than faster targets in this list. Insert those deleted targets in the list \emph{Eliminated}. Don't remove targets which move towards the other direction so they are crossing the origin\\

\State Rejoin \emph{Left, Right, Top and Bottom} for the \emph{Main Algorithm}
\end{algorithmic}
\end{algorithm}
\end{minipage}

\begin{minipage}{1\linewidth}
\begin{algorithm}[H]
\begin{algorithmic}

\State \textbf{Main Algorithm}
\State Let $time$ be the time-array in which a target $t_i$ is intercepted
\State Let $current$ be the target the pursuer just intercepted
\State Let \emph{OUTPUT} be the list of intercepted targets 
\State $current\leftarrow$ origin (initial position of the pursuer)
\State $OUTPUT.add(current)$ 
\State $PrioQ$ is a priorityqueue which sorts the targets in order of nonincreasing priority\\

\For {\textbf{each} $t_i \in T$}
\State $time[t_i]\leftarrow \infty$
\State $PrioQ.add(t_i)$
\EndFor

\While {$PrioQ$ is not empty}
\If {all remaining targets in $PrioQ$ are located on one side of the intersection,\\~~~ once the pursuer reached the intersection from \emph{current}}
\State Calculate the time required to intercept all remaining targets (and return 
\State to the origin), sort them in order of nondecreasing interception times and 
\State add all to $OUTPUT$
\State Go to the \emph{Postprocessing} step
\EndIf \\

\State Calculate priority of $t_i\in PrioQ$ \textcolor{red}{TODO: Find a suitable priority measure}
\State Update $PrioQ$
\State $prev \leftarrow current$
\State $current \leftarrow prioQ.poll()$
\State $time[current] \leftarrow time[prev] + \pi[prev\rightarrow current]$
\State Update the position of each $t_i\in PrioQ$
\State $Intercepted \leftarrow$ intercepted targets between prev and current (also add $current$)
\State Sort $Intercepted$ in order of nondecreasing interception times and add all to $OUTPUT$
\EndWhile

\end{algorithmic}
\end{algorithm}
\end{minipage}

\pagebreak

\begin{minipage}{1\linewidth}
\begin{algorithm}[H]
\begin{algorithmic}

\State \textbf{Postprocessing}
\For {\textbf{each} $t_i \in T$}
\If {a target $e_j \in Eliminated$ is intercepted between $t_i$ and $t_{i+1}$}
\State Calculate the interception time of $e_j$ and add $e_j$ at the correct position in $OUTPUT$
\State Remove $e_j$ from $Eliminated$ 
\EndIf
\EndFor
\State \textbf{return} $OUTPUT$ with the respective interception time

\end{algorithmic}
\end{algorithm}
\end{minipage}

%\section*{Laufzeit}





\end{document}