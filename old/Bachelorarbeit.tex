\documentclass{acm_proc_article-sp}
\usepackage[english]{babel}
\usepackage[utf8]{inputenc}
\usepackage[T1]{fontenc}
\usepackage{graphicx} 			%Grafiken
\usepackage{xcolor}   			%Farbige Schrift ermöglichen
\usepackage{amsmath}  			%Zusätzliche Mathebefehle

% own packages
%--------------------------------------------------------------------------------------------------------
\usepackage{hyperref}			

\begin{document}

\title{Algorithms for the Moving Target Travelling Salesmen Problem}
\subtitle{[Bachelor thesis]}
\numberofauthors{1} 
\author{
\alignauthor
Felix Greuling\\
       \affaddr{Universität zu Lübeck}\\ 
       \affaddr{Ratzeburger Allee 160}\\
       \affaddr{23562 Lübeck, Deutschland}\\ 
       \email{Felix.Greuling@student.uni-luebeck.de}
}
\date{01. Juni 2018}
\maketitle

%Abstract
\begin{abstract}
Lorem ipsum dolor sit amet, consetetur sadipscing elitr, sed diam nonumy eirmod tempor invidunt ut labore et dolore magna aliquyam erat, sed diam voluptua.
\end{abstract}

%------------------------------------------------------------------------------------------
\section{Introduction}
Lorem ipsum dolor sit amet, consetetur sadipscing elitr, sed diam nonumy eirmod tempor invidunt ut labore et dolore magna aliquyam erat, sed diam voluptua.

%------------------------------------------------------------------------------------------
\section{Definitions}
In MT-TSP we consider an amount of targets $T=\{t_1,...,t_n\}$ and a set of velocities $V=\{v_1,...,v_n\}$ so that each moving with a constant movement speed $\overrightarrow{v_i}$. A pursuer starts at the origin $O$ (a defined position), moving with a velocity of $v_p$. His aim is to visit all targets once and finishes with returning to the origin.

Therefore, we can model this problem as the following graph \\
$G=(T, V, O, v_p)$. 

%------------------------------------------------------------------------------------------
\section{Instances of Moving-Target TSP}

It was proven that MT-TSP is NP-hard. Some instances can result in an unbounded error, whenever the pursuer choses a non-optimal tour. Therefore, the condition $v_p > |\overrightarrow{v_i}|$ must apply, to avoid these errors. This was proven by the authors in \cite{helvig}, since the goal is the most fast optimal tour.  Thus, instead of directly calculating the tour of the pursuer, it is necessary to determine the solvability of the input. Whenever it is not possible we gain a 'No'-instance, otherwise we go ahead and calculate the tour. \\
This paper presents two concrete cases:
\begin{itemize}
\item[1)]
1D-case: 
Each target's movement is limited to a single line 

\item[2)]
2D-case:
The movement direction of a target consists of a two-dimensional vector 

\end{itemize}

Each case is investigated for the shortest and the fastest tour. Helvig , Robins and Zelikovsky presented in \cite{helvig} the first heuristics in the field of MT-TSP. However, determining the approximation for general cases is hard, because there are many influences that determine the complexity of the problem. The approximation research in \cite{hammar} showed that MT-TSP cannot be approximated better than by a factor of $2^{\pi (\sqrt{n})}$ by a polynomial time algorithm unless P=NP.

Later on, we will compare new heuristics and algorithms with those that recently proposed in \cite{moraes}. This paper examines Genetic Algorithm, Simulated Annealing and Ant Colony Optimization to solve MT-TSP.


%------------------------------------------------------------------------------------------
\subsection{One-dimensional-case}

This specific case is already introduced extensive in \cite{helvig}. Each movement is fixed on a single line, thus, there are two possible directions for a target and the pursuer. 


%Furthermore, the case of the shortest tour is introduced, which was not examined in the work of %Helvig, Robins and Zelikovsky.

%------------------------------------------------------------------------------------------
\subsection{Two-dimensional-case}

In this case another dimension is added. We will consider different scenarios with fixed numbers of targets and corridors. Compared to the 1D-case, in which a target is fixed on a line, targets freely move within the space. Therefore, we define corridors, representing the number of vectors in which the targets move, resulting in different complexities.

This paper focusses on the 2D-case. Previous works did not construct concrete heuristics for this type of case, therefore, this paper shows new aspects in the field of MT-TSP.

%------------------------------------------------------------------------------------------
\subsection{Input \& Output}
To set up heuristics a suitable input and output are necessary. Thus, the MT-TSP can be modelled as graph problem (refering section 2), further heuristics just expect a graph $G$ as parameter. 

The Output, on the other hand, can look very different. The most obvious parameter are the length and the time needed to finish the tour. Whenever, there is no possible tour, an MT-TSP algorithm needs to return a 'No'-instance. 

Furthermore, a tour should be comprehensible. Therefore, the targets are displayed in the order in that the pursuer executes the tour. For each target the initial coordinates and the time and the position whenever the pursuer intercepts the target are monitored. To fully understand and analyse the tour a visualization is a good application, but not necessary. 

%------------------------------------------------------------------------------------------
\section{Algorithms for MT-TSP}
Lorem ipsum dolor sit amet, consetetur sadipscing elitr, sed diam nonumy eirmod tempor invidunt ut labore et dolore magna aliquyam erat, sed diam voluptua.

\subsection{Algorithms for 1D-case}
Lorem ipsum dolor sit amet, consetetur sadipscing elitr, sed diam nonumy eirmod tempor invidunt ut labore et dolore magna aliquyam erat, sed diam voluptua.


\subsubsection{Algorithm by Helvig, Robins and Zelikovsky}
As previously mentioned, the authors of \cite{helvig} showed an exact $\mathcal{O}(n^2)$-time algorithm for 1D-cases, which is based on dynamic programming. The goal is to receive the fastest, optimal tour, intercepting each target. 

The algorithm is based on the idea, that the pursuer cannot change its direction until it intercepts the fastest target ahead of him. Therefore, only targets, where the pursuer changes direction, need to be considered. Thus, the tour is a sequence of snapshots with turning points and a snapshot can be denoted as a state $A$. Each tour always starts in $A_0$ and finishes in $A_{end}$. A state $A_i=(s_k, s_f)$ consists of a target which just intercepted ($s_k$) and the fastest target on the other side of the origin ($s_f$). Ensuing a state, there are at most two transitions to other states, the fastest on the left or the right of the pursuer. The time of a transition $\tau$ is determined with $t(\tau)$. 

The algorithm begins with dividing all targets into two lists, according to whether a target is on the left or right side of the origin. Afterwards, both lists are sorted in descending order of their speed. The algorithm executes at each state with one of the following steps:
\begin{itemize}
\item
if the state has no transitions into it, we proceed to the next state in the list
\item
if the pursuer has intercepted all of the targets on one side, we make a transition into
the final state $A_{final}$
\item
we make two transitions which correspond to sending the pursuer after either the
fastest target on the left or the fastest target on the right

\end{itemize}

To denote the time used to reach a state $B_i$ with the shortest sequence of states, the time-function $t$ is used with $t(A_i)=min\{t(A)+t(\tau)|\tau:A\rightarrow B\}$. As usual in dynamic programming, we can traverse backwards from $A_{final}$ to $A_0$, which denotes the reversed list of turning points that describes an optimal tour.

\subsubsection{Results of Helvig, Robins and Zelikovsky}

These first heuristics in the field of MT-TSP investigate three cases, which result in different approximations. three concrete cases: $\mathcal{O}(n^2)$-time in 1D-case, $(1+\alpha)$-approximation, when the number of moving targets is small and $\mathcal{O}(n^2)$-time, whenever enough targets are stationary. 

However, the authors focused on minimizing the time a pursuer needs to perform the tour. The heuristics just considered 1D-cases and did not focus on finding the smallest tour.



%------------------------------------------------------------------------------------------
\subsection{Algorithms for 2D-case}
Lorem ipsum dolor sit amet, consetetur sadipscing elitr, sed diam nonumy eirmod tempor invidunt ut labore et dolore magna aliquyam erat, sed diam voluptua

%------------------------------------------------------------------------------------------
\subsection{Algorithms for general cases}
Most of previous works refer to general cases without specific consideration of 1D- and 2D-cases. In the following subsection innovative approaches are shown, just developed in 2019, mentionned in \cite{moraes}.

%------------------------------------------------------------------------------------------
\subsubsection{Algorithm by de Moraes and de Freitas}
In \cite{moraes} the authors proposed to solve MT-TSP with colony-algorithms. This includes the Genetic Algorithm (GA), Simulated Annealing (SA) and Ant Colony Optimization (ACO), which try to evolve a population to an optimum.

GA evolve candidate solutions until a certain fitness or a number of iterations is reached. The operands selection, crossover and mutation take care of generating new variations within the population. 

\textcolor{red}{TODO: GA, SA, ACO description}

%------------------------------------------------------------------------------------------
\section{Summary and Outlook}
Lorem ipsum dolor sit amet, consetetur sadipscing elitr, sed diam nonumy eirmod tempor invidunt ut labore et dolore magna aliquyam erat, sed diam voluptua.
%\cite{moraes}
%\cite{helvig}

\section{Fragen}
\begin{itemize}
\item
$(1+\alpha)$-approximation algorithm, where $\alpha$ denotes the approximation ratio of the best classical TSP heuristic

\item
It was shown in [4] that the (2 - $\epsilon$)-approximation is NP-hard even in the case when only two targets are
moving.

\end{itemize}

\bibliography{bibtex}{}
\bibliographystyle{plain}
\end{document}