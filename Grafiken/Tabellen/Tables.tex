\documentclass{scrartcl}
\usepackage{enumitem}
\RequirePackage[utf8]{inputenc}
\RequirePackage[ngerman]{babel}
\RequirePackage[T1]{fontenc}
\RequirePackage{lmodern}
\usepackage{graphicx}
\usepackage{amsmath}

\usepackage{listings}
\usepackage{subfig}
\usepackage{tikz}
\usetikzlibrary{positioning, shapes, snakes}
\usepackage{array,ragged2e}

\usepackage{typearea}
\areaset[5mm]{135mm}{237mm}

\tikzstyle{red} = [color=red, circle, draw]
\tikzstyle{blue} = [color=blue, circle, draw]
\tikzstyle{gray} = [color=gray, circle, draw]
\tikzstyle{green} = [color=green, circle, draw]

\begin{document}

%$n$& Testinstanzen & Durchschnitt betrachtete Knoten & Durchschnitt Verhältnis betrachtete Knoten:Gesamtknoten & Durchschnitt Schnitte & Durchschnitt erreichte Wurzeln & Fails\\

\begin{table}[htpb]
\centering
\scalebox{0.94}{
\begin{tabular}{ccccccc} 
\hline
$n$ &Instanzen& betr. Knoten & Anteil Knoten & Schnitte & ber. Blätter & Fails\\ \hline
1& 10000& ~~~~~~1.00& 1.000000& ~~~~~~0.00& ~~~1.00& 0 \\
2& 10000& ~~~~~~3.75& 0.939151& ~~~~~~0.24& ~~~1.57& 0 \\
3& 10000& ~~~~~11.75& 0.783513& ~~~~~~0.95& ~~~3.14& 0 \\
4& 10000& ~~~~~37.81& 0.590758& ~~~~~~3.22& ~~~6.71& 0 \\
5& 10000& ~~~~130.44& 0.401344& ~~~~~11.87& ~~15.02& 0 \\
6& 10000& ~~~~471.76& 0.241184& ~~~~~47.23& ~~32.46& 0 \\
7& 10000& ~~~1819.94& 0.132852& ~~~~194.96& ~70.33& 0 \\
8& 10000& ~~~7353.01& 0.067090& ~~~~852.13& ~150.30& 0 \\
9& 10000& ~~31426.98& 0.031860& ~~~3833.61& ~313.76& 0 \\
10&~1000& ~140342.54& 0.014228& ~~18169.79& ~620.86& 0 \\
11&~1000& ~650570.40& 0.005996& ~~88862.13& 1304.56& 0 \\
12&~~100& 3562601.51& 0.002736& ~457589.77& 3246.11& 0 \\
13&~~100& 8184931.87& 0.000484& 1523396.29& 4303.92& 22\\
14&~~~10& 5033246.67& 0.000021& 1147070.33& 2025.67& 7 \\ \hline
\end{tabular}}
\caption{blub}
\label{tab:ExpBF1}
\end{table}

\begin{table}[htpb]
\centering
\scalebox{1}{
\begin{tabular}{ccccccc} 
\hline
$n$ & Instanzen& $max\{v_i\}$ & D Güte & max Güte & opt. Touren & Anteil opt.\\ \hline
~8& 10000& 20& 1.10& ~~2.59& 6313& 0.63 \\
~8& 10000& 40& 1.41& ~11.93& 3341& 0.33 \\
~8& 10000& 60& 3.53& 328.88& 1390& 0.14 \\
10& ~1000& 20& 1.14& ~~2.15& ~518& 0.52 \\
10& ~1000& 40& 1.54& ~~9.22& ~220& 0.22 \\
10& ~1000& 60& 5.32& 282.26& ~~83& 0.08 \\
12& ~~100& 20& 1.16& ~~2.31& ~~43& 0.43 \\
12& ~~100& 40& 1.56& ~~4.79& ~~12& 0.12 \\
12& ~~100& 60& 5.30& ~82.29& ~~~5& 0.05 \\
 \hline
\end{tabular}}
\caption{blub}
\label{tab:ExpBF2}
\end{table}

\begin{table}[htpb]
\centering
\scalebox{1}{
\begin{tabular}{cccc} 
\hline
$n$ & $v_i$& $v_{\kappa}$ & Tourlänge\\ \hline
~~5& 20 & 200 & ~~~~~~~~~~~~~~4.11 \\
~~5& 40 & 100 & ~~~~~~~~~~~~15.59 \\
~~5& 60 & ~61 & ~~~~~7198012.00 \\
~25& 20 & 200 & ~~~~~~~~~~~~12.75 \\
~25& 40 & 100 & ~~~~~~~~~~~136.28 \\
~25& 60 & ~61 & 1199646496.57 \\
~50& 20 & 200 & ~~~~~~~~~~~~13.46 \\
~50& 40 & 100 & ~~~~~~~~~~~149.14 \\
~50& 60 & ~61 & 1622776017.58 \\
~75& 20 & 200 & ~~~~~~~~~~~~17.28 \\
~75& 40 & 100 & ~~~~~~~~~~~168.81 \\
~75& 60 & ~61 & 1747321267.83 \\
100& 20 & 200 & ~~~~~~~~~~~~14.17 \\
100& 40 & 100 & ~~~~~~~~~~~164.75 \\
100& 60 & ~61 & 1615929567.82 \\
 \hline
\end{tabular}}
\caption{blub}
\label{tab:ExpBF3}
\end{table}

\begin{table}[htpb]
\centering
\scalebox{1}{
\begin{tabular}{cccc} 
\hline
$n$ & |Koordinatenlimit| & $\omega$\\ \hline
~8& ~~~~100& $\{64,21,19\}$ \\
~8&  10.000& $\{54,17,40\}$ \\
10& ~~~~100& $\{98,39,74\}$ \\
10&  10.000& ~~~$\{75,3,6\}$ \\
12& ~~~~100& ~$\{69,14,8\}$ \\
12&  10.000& ~~~$\{24,4,7\}$ \\
\hline
\end{tabular}}
\caption{Der jeweils beste Gewichte-Kandidat von $1.000$ zufälligen Gewichten für $100$ Instanzen.}
\label{tab:ExpGewichte}
\end{table}

\begin{figure}[htpb]
\centering
\includegraphics[scale=0.24]{../Verwendete/Exp1D_2.png}
\caption{blub}
\label{bla}
\end{figure}

\begin{table}[htpb]
\centering
\footnotesize
\scalebox{0.9}{
\begin{tabular}{>{\scriptsize}lll} \hline
Parameter & Datentyp & Bedeutung/Belegung \\ \hline
&& Welcher Fall/Instanz soll berechnet werden? \\
IS\_1D & boolean                          & true: 1D-Fall \\ 
&& false: 2OA-Fall \\\hline
&& Soll eine zufällige Eingabe generiert werden?\\
GENERATE\_NEW\_INPUT & boolean            & true: Ja \\ 
&& false: Eigene Eingabe-Datei verwenden \\\hline
INPUT\_FILE\_1D & String                  & 1D-Instanz-Datei \\ \hline
INPUT\_FILE\_2OA & String                 & 2OA-Instanz-Datei \\ \hline
USE\_BRUTE\_FORCE & boolean               & Brute-Force-Algorithmus verwenden? \footnotemark \\ \hline
PRINT\_TARGETS\_AND\_STEPS & boolean      & Sollen die Schritte angezeigt werden? \\ \hline
N & int                                   & Eingabegröße \\ \hline
PURSUER\_POS & int                        & Startposition Verfolger / Ursprung \\ \hline
V\_PURSUER & int                          & Verfolgergeschwindigkeit \\ \hline
V\_I\_MAX & int                           & Maximale Geschwindigkeit eines Ziels \\ \hline
LIMIT\_COORD & int                        & Maximale Startkoordinate eines Ziels \\ \hline
&& Auf welcher Achse soll der Verfolger starten?\\
PURSUER\_POS\_AXIS & boolean              & true: waagerechte Achse \\ 
&& false: senkrechte Achse \\ \hline
WEIGHTS & int[]                           & Gewichte-Array der Größe 3  \\ \hline
&& Feste für die Geschwindigkeiten aller Ziele?\\
CONSTANT\_VELOCITIES & boolean            &  true: Ja \\
&&False: zufällige Geschwindigkeit \\ \hline
TESTING\_ENABLED & boolean                & Soll eine der Testmethoden verwendet werden? \\hline
TEST\_SPECIFIC\_SEQUENCE & boolean        & Vorgegebene Sequenz an Zielen berechnen?\\ \hline
TEST\_MULTIPLE\_PRIO\_INSTANCES & boolean & Gewichte mit zufälligen Instanzen testen?\\ \hline
TEST\_MULTIPLE\_BF\_INSTANCES & boolean   & Bäume mit zufälligen Instanzen testen? \\ \hline 
TEST\_WEIGHTS & boolean                   & Zufällige Gewichte für eine Instanz testen? \\ \hline
TEST\_PRIO\_QUALITY & boolean             & Güte für mehrere Instanzen testen? \\ \hline
TEST\_ITERATIONS  & int                   & Anzahl an Testiterationen \\ \hline
TEST\_INSTANCES & int                     & Anzahl an generierten Testinputs \\ \hline
LIMIT\_RANDOM\_WEIGHTS & int              & Maximaler, zufälliger Gewichtswert \\ \hline
\end{tabular}}
\caption{Belegungen der Parameter in \emph{Constants.java} und deren Bedeutung.}
\label{Constants}
\end{table}
\footnotetext{Bei solchen boolean-Parametern gilt: true: Ja, false: Nein}

\end{document}

%\draw[very thick] (-6,0) -- (6,0);
%\draw (0,0) -- (0,2); 