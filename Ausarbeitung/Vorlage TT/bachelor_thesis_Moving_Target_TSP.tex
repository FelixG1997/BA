\documentclass[english,version-2019-07]{uzl-thesis}


% Copy this file as a template for your thesis. You will have to take
% action at all places marked by
%
% !!!!!!!!!!!!!!!!!!!!!!!!!!!!!!!!!!
% !!! Your action is needed here !!!
% !!!!!!!!!!!!!!!!!!!!!!!!!!!!!!!!!!
%
% The first place your action is needed is the first line of this
% document:
%
%
% Language of the thesis:
%
% You must use either 'german' or 'english' above, depending on the
% language used in the main text. This will automatically setup a lot
% of things in the background.
%
%
% Version of the class:
%
% You must specify which version of the thesis class is to be
% used. This is important in case the class style changes in later
% years, but we still want an older thesis to look the same, even when
% things are changed in the class.
%
% Do not change or remove the version-xxxx key.
%
%
% Text encoding:
%
% Your thesis *must* be encoded in utf8 (unicode), which is the
% default in most editors these days. Do *not* change this to latin8.



%%%
%
% Main setup:
%
%%%
%
% You must use the \UzLThesisSetup command to specify numerous things
% about your thesis. This includes the entries on the title page, the 
% abstracts, and the bibliography style. You do so by specifying
% so-called "values" for so-called "keys". For instance, 
% for the key "Autor" you must provide your name as the value. You do
% so by writing 'Autor = {Max Mustermann}', that is, the value is put
% into curly braces. You can use the \UzLThesisSetup command
% repeatedly and the order in which you provide the keys is not
% important. 
%
% Everything shown on the title page must be in German -- even
% if the thesis is written in English! Just insert German text for
% German keys and English text for English keys (like 'Abstract' needs
% English text, while 'Zusammenfassung' needs German text).

\UzLThesisSetup{
  %
  % !!!!!!!!!!!!!!!!!!!!!!!!!!!!!!!!!!
  % !!! Your action is needed here !!!
  % !!!!!!!!!!!!!!!!!!!!!!!!!!!!!!!!!!
  %
  % First, specify the institut or clinic at which the thesis was
  % written. You get the logo file from them (make sure it has the
  % correct size, namely the same as the example). If they do not have
  % a logo, the university's default logo is used.
  %
  % The 'verfasst' gets two arguments. Change the first to {an der}
  % for clinics, as in 'Verfasst = {an der}{Medizinischen Klinik I}'
  %
  Logo-Dateiname        = {uzl-thesis-logo-itcs.pdf},
  Verfasst              = {am}{Institut für Theoretische Informatik},
  %
  % The titles:
  %
  Titel auf Deutsch     = {
    Algorithmen für bewegende Ziele in TSP
  }, 
  Titel auf Englisch    = {
    Algorithms for Moving-Target TSP
  },
  %
  % Author and supervisor:
  % 
  % Note that the 'Betreuer' or 'Betreuerin' is the supervisor, that
  % is, the professor who officially supervises the thesis. If there
  % is also an assistent of the professor who helped (typically a
  % lot), use 'Mit Unterstützung von' to thank that person. If the
  % thesis was mainly written 'externally' at some company or another
  % institute, point this out using 'Weitere Unterstützung'. 
  % 
  % For your own name, do *not* add things like "BSc" or "BSc
  % cand.". For the supervisor, you should normally include
  % "Prof. Dr." or "PD Dr." (ask your supervisor, what is
  % appropriate), but nothing more (so no
  % "Univ.-Prof. Dr. Dr. h.c. mult." unless your supervisor insists).  
  %
  Autor                 = {Felix Greuling},
  Betreuerin            = {Prof. Dr. Maciej Liskiewicz},
  % 
  % Optional: Supporting persons and institutions. The text should be
  % in German, even for an English thesis.
  %
  %Mit Unterstützung von = {Harry Hilfreich},
  % 
  %   Weitere Unterstützung = {
  %     Die Arbeit ist im Rahmen einer Tätigkeit bei der Firma Muster GmbH
  %     entstanden.
  %   },
  %
  %
  % Your Degree Programm (Studiengang)
  %
  % Specify 'Bachelorarbeit' or 'Masterarbeit' and the degree
  % programme. Make sure the name of programme is correct and not
  % some abbreviation or some incorrect variant. For instance:
  % 'Medizinische Ingenierwissenschaft', but not 'MIW';
  % 'Medizinische Informatik', but not 'Medizin-Informatik';
  % 'Informatik', but not 'Informatik (SSE)'.
  %
  % Use German names for German programmes and English names for
  % English ones, so 'Infection Biology', not 'Infektionsbiologie'. 
  % For programmes that have a German bachelor and an English master,
  % use the German name for a bachelor thesis and the English name for
  % the master thesis.
  %
  Bachelorarbeit,
  Studiengang           = {Informatik},
  %
  % Date on which the thesis is turned in:
  %
  Datum                 = {28. November 2019},
  %
  % The English abstract. You must always provide abstracts in German
  % and in English. 
  %
  Abstract              = {
    It is not easy to write a thesis that does not only advance
    science, but that is also a pleasure to read. While the scientific
    contribution of a thesis is undoubtedly of greater importance, the
    impact of \emph{writing well} should not be underestimated: If
    the person who grades a thesis finds no pleasure in the reading,
    that person are also unlikely to find pleasure in giving outstanding
    grades. A well-written text uses good German or English phrasing with a clear and correct 
    sentence structure and language rhythm, there are no spelling
    mistakes and the author's arguments are presented in a
    clear, logical and understandable manner using well-chosen
    examples and explanations. In addition, a nice-to-read font and a
    pleasing layout are also helpful. The \LaTeX\ class presented in
    this document helps with the latter: It contains a number of
    ready-to-use designs and 
    takes care of many small typographical chores.
  },
  Zusammenfassung       = {
    Es ist nicht leicht, eine Abschlussarbeit so zu schreiben, dass sie
    nicht nur inhaltlich gut ist, sondern es auch eine Freude ist, sie
    zu lesen. Diese Freude ist aber wichtig: Wenn die Person, die die 
    Arbeit benoten soll, wenig Gefallen am Lesen der Arbeit findet,
    so wird sie auch wenig Gefallen an einer guten Note
    finden. Glücklicherweise gibt es einige Kniffe, gut lesbare
    Arbeiten zu schreiben. Am wichtigsten ist zweifelsohne, dass
    die Arbeit in gutem Deutsch oder Englisch verfasst wurde mit klarem
    Satzbau und gutem Sprachrhythmus, dass keine Rechtschreib- oder
    Grammatikfehlern im Text auftauchen und dass die Argumente der
    Autorin oder des Autors klar, logisch, verständlich und gut
    veranschaulicht dargestellt werden. Daneben sind aber auch gut
    lesbare Schriftbilder und ein angenehmes Layout hilfreich. Die Nutzung
    dieser \LaTeX-Vorlage hilft der Schreiberin oder dem Schreiber
    dabei zumindest bei Letzterem: Sie umfasst gute, sofort nutzbare
    Designs und sie kümmert sich um viele typographische
    Details.  
  },
  %
  % Optional: 'Danksagungen' (German) or 'Acknowledgements'
  % (English). Both keys are optional and both have the same effect of
  % adding an acknowledgements text after the abstracts and before the
  % table of contents.
  %
  Acknowledgements      = {
    This is the place where you can thank people and institutions, do
    not try to do this on the title page. The only exception is in
    case you wrote your thesis while working or staying at a company or abroad. Then you
    should use the \Latex{Weitere Unterstützung} key to provide a text
    (in German) that acknowledges the company or foreign
    institute. For instance, you could use texts like »Die Arbeit
      ist im Rahmen einer Tätigkeit bei der Firma Muster GmbH
      entstanden« or »Die Arbeit ist im Rahmen eines
      Forschungsaufenthalts beim Institut für Dieses und Jenes an der
      Universität Entenhausen entstanden«. Do not name and thank
      individual persons from the company or foreign institute on the
      title page, do that here. 
  },
  % Bibliography style: Choose between
  % 
  % 'Alphabetische Bibliographie'
  % for all degree programmes in the natural sciences 
  % 
  % 'Numerische Bibliographie'
  % alternative for all other degree programmes
  % 
  % Either will load biblatex and setup the citation methods and the
  % bibliography styles correctly. You should not mess with them.
  % 
  Alphabetische Bibliographie
  % Alternatively:
  % Numerische Bibliographie
}


\UzLStyle{alegrya modern design}
%
% A design that uses the sans serif version of the Alegrya font for
% the headlines. This is a nice modern overall design.
%
%%%




%%%%%%%%
%
% Now, include the package you need here using \usepackage. 
%
% However, many standard packages are already loaded by the class:
%
% amsmath, amssymb, amsthm, babel, biblatex, csquotes, etoolbox,
% filecontents, fontspec, geometry, hyperref, tikz (with libraries
% arrows.meta, positioning and shapes), varioref, url 
%
% Indeed, in many cases you will not need any extra packages.
%
%%%%%%%





\begin{document}

%
% The title page and table of contents will be inserted automatically
% here. 
%



\chapter{Introduction}
% In a German thesis write: \chapter{Einleitung}



% !!!!!!!!!!!!!!!!!!!!!!!!!!!!!!!!!!
% !!! Your action is needed here !!!
% !!!!!!!!!!!!!!!!!!!!!!!!!!!!!!!!!!
%
% Replace with your own introduction:

\lstMakeShortInline[style=code,style=inline,language={[LaTeX]tex},moretexcs={chapter}]|
 


\section{Contributions of this Thesis}
% In a German thesis write: \section{Beiträge dieser Arbeit}

% !!!!!!!!!!!!!!!!!!!!!!!!!!!!!!!!!!
% !!! Your action is needed here !!!
% !!!!!!!!!!!!!!!!!!!!!!!!!!!!!!!!!!
%
% Replace with a detailed account of your contributions:



\section{Related Work}
% In a German thesis write: \section{Verwandte Arbeiten}

% !!!!!!!!!!!!!!!!!!!!!!!!!!!!!!!!!!
% !!! Your action is needed here !!!
% !!!!!!!!!!!!!!!!!!!!!!!!!!!!!!!!!!
%
% Replace with a detailed account of what other people have already
% researched concerning your thesis's theme. Even when (indeed,
% especially when) there has been only little or even no research by
% other people, you should explain in detail that this is the case and
% why it is the case. 



\section{Structure of this Thesis}
% In a German thesis write: \section{Aufbau dieser Arbeit}


% !!!!!!!!!!!!!!!!!!!!!!!!!!!!!!!!!!
% !!! Your action is needed here !!!
% !!!!!!!!!!!!!!!!!!!!!!!!!!!!!!!!!!
%
% Replace the following by one or two paragraphs describing the
% thesis's structure.






% !!!!!!!!!!!!!!!!!!!!!!!!!!!!!!!!!!
% !!! Your action is needed here !!!
% !!!!!!!!!!!!!!!!!!!!!!!!!!!!!!!!!!
%
% Replace the whole text chapter with the main text of your thesis! 

\chapter{Guide to the Thesis Class}%: Document Setup and Document Structure}
\label{chapter-use}




\chapter{Conclusion}
% In a German thesis write: \subsection{Zusammenfassung und Ausblick}


% !!!!!!!!!!!!!!!!!!!!!!!!!!!!!!!!!!
% !!! Your action is needed here !!!
% !!!!!!!!!!!!!!!!!!!!!!!!!!!!!!!!!!
%
% Replace the following with your conclusion



This template document got much longer than I had initially intended
with more and more hints and comments becoming part of the text. The
reason is, of course, that writing a thesis is not easy since there
are a \emph{lot} of things to consider. However, you have six months
to write your thesis, so you stand a decent chance to get most things 
right.

Do some great scientific research now and report on it in a thesis
that is a pleasure to read. 




% Normally, the bibliography comes next at this point. Do *not* (try
% to) include further indices and tables like an index or
% a list of figures or a list of tables or such things. Nobody
% actually uses them and they just use up space. 
%
% You *can* however include a glossary, if this seems appropriate. It
% goes here as an unnumbered chapter. Most thesis will *not* need a
% glossary: a well-written text (re)explains strange words and
% concepts as necessary. However, there are situations where a
% glossary may be helpful.














%%%
% 
% Bibliographies
%
%%%
%
% The uzl-thesis class will load biblatex for the bibliography
% management. This is a powerful package, see its documentation for
% details. The styles will be setup correctly and automatically by
% choosing one of the two style keys as described earlier.
%
% In order for the bibliography to work, run latex in the following
% order (which is the standard order):
% 
% > lualatex thesis-example
% > bibtex thesis-example
% > lualatex thesis-example
% 
% Add BibTeX files using \addbibresource or use the {bibtex entries}
% environment (see below).
%
%%%
%
% Although everyting is normally setup automatically, you can change
% the options passed to biblatex using the key 'biblatex';
% for instance,
%
%   \UzLThesisSetup{biblatex={firstinits=false}}
%
% will switch off shortened first names. Normally, you will not need
% this key in your preamble. 
% 
% Note that the bibtex program is used as the 'backend' of biblatex
% by default (rather than biber, which is the preferred program of
% biblatex). This means that you can (and must) run *bibtex* after you
% have run lualatex on your thesis. If you wish to use biber instead
% of bibtex, say 'biblatex={backend=biber}'. 
% 
%%%
%
% The following environment is optional. It allows you to keep the
% bibtex entries for your thesis right here in the thesis file. What
% happens is that each time this tex file is processed, the contents
% of the following environment gets written to the file
% \jobname-bibtex-entries.bib (this file gets overwritten each
% time). Independently, \addbibresource{\jobname-bibtex-entries.bib}
% is always called if the file \jobname-bibtex-entries.bib
% exists. 
%
% In result, you can edit and keep the bibliography's bibtex entries
% right here. If you change something here, run latex, then bibtex,
% then latex once more.
%
% If you would like to manage the bibtex entries in a separate file,
% remove the below environment, delete the \jobname-bibtex-entries.bib
% file and instead write
%
% \addbibresource{filename-of-your-bibtex-file.bib}
%
% in the preamble.
%
%%%


% !!!!!!!!!!!!!!!!!!!!!!!!!!!!!!!!!!
% !!! Your action is needed here !!!
% !!!!!!!!!!!!!!!!!!!!!!!!!!!!!!!!!!
%
% Replace following example entries with the ones of your thesis.

\begin{bibtex-entries}

@article{helvig,
  title={The moving-target traveling salesman problem},
  author={Helvig, Christopher S and Robins, Gabriel and Zelikovsky, Alex},
  journal={Journal of Algorithms},
  volume={49},
  number={1},
  pages={153--174},
  year={2003},
  publisher={Elsevier}
}

@article{moraes,
  title={Experimental Analysis of Heuristic Solutions for the Moving Target Traveling Salesman Problem Applied to a Moving Targets Monitoring System},
  author={de Moraes, Rodrigo S and de Freitas, Edison P},
  journal={Expert Systems with Applications},
  year={2019},
  publisher={Elsevier}
}

@inproceedings{hammar,
  title={Approximation results for kinetic variants of TSP},
  author={Hammar, Mikael and Nilsson, Bengt J},
  booktitle={International Colloquium on Automata, Languages, and Programming},
  pages={392--401},
  year={1999},
  organization={Springer}
}


\end{bibtex-entries}
\end{document}