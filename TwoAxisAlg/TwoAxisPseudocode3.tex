\documentclass{scrartcl}


% input encoding
\usepackage[utf8]{inputenc}

% new german spelling
\usepackage[ngerman]{babel}

% choose font
\usepackage[T1]{fontenc}
\usepackage{lmodern}

% KOMA-Script options
\KOMAoptions{%
  parskip=full,%
  fontsize=12pt,%
  DIV=calc}

% page options
\usepackage{geometry}
\geometry{a4paper, portrait, left=2.5cm, right=2cm, top=2cm, bottom=2cm, includefoot}

% color and images
\usepackage{xcolor}
\usepackage{graphicx}

% draw with tikz
\usepackage{tikz}
\usetikzlibrary{positioning,automata}

% quotes
\usepackage[german=guillemets]{csquotes}

% better enumerations
\usepackage{enumitem}

% title spacing fix
%\usepackage{titlesec}
%\titlespacing*{\subsection}{0pt}{2\baselineskip}{-0.5\baselineskip}
\RedeclareSectionCommand[
  beforeskip=-2\baselineskip,
  afterskip=.5\baselineskip]{section}
\RedeclareSectionCommand[
  beforeskip=-1.5\baselineskip,
  afterskip=.1\baselineskip]{subsection}
\RedeclareSectionCommand[
  beforeskip=-1\baselineskip,
  afterskip=.05\baselineskip]{subsubsection}
\RedeclareSectionCommand[
  beforeskip=.5\baselineskip,
  afterskip=-1em]{paragraph}
\RedeclareSectionCommand[
  beforeskip=-.5\baselineskip,
  afterskip=-1em]{subparagraph}

% math
\usepackage{amsmath}
\usepackage{amssymb}
\usepackage{amsthm}

% custom theorems
\usepackage{thmtools}
\declaretheorem[style=definition]{aufgabe}

% set special behaviour for hyperlinks in pdfs
\usepackage[breaklinks=true]{hyperref}

% display program code
\usepackage{listings}

\lstset{ %
  backgroundcolor=\color{white!98!black},  
  basicstyle=\ttfamily\footnotesize,
  breakatwhitespace=true,
  breaklines=true,
  commentstyle=\color{green!40!black},
  frame=single,
  keywordstyle=\color{blue!70!black},
  language=ML,
  numbers=left,
  numbersep=8pt,
  numberstyle=\scriptsize\color{black!80}, 
  rulecolor=\color{black!70},
  stepnumber=1,
  stringstyle=\color{purple!80!black},
  tabsize=2,
  title=\lstname,
  language=ML,
  belowskip=-0.8\baselineskip,
  aboveskip=-0.8 \baselineskip,
}

% Pseudocode
\usepackage{algorithm}
\usepackage{algpseudocode}

% positioning
\usepackage{float}

% document information
\title{Moving-Target TSP in two-orthogonal-axes}
\subtitle{Pseudocode}
\author{Felix Greuling (666020)}
\date{21. Oktober 2019}

\begin{document}
\maketitle

% if, state, for, endif, endfor, else
% \Procedure{CalculateFunSums}{$T, w, visited, V$}
\begin{minipage}{1\linewidth}
\begin{algorithm}[H]
\begin{algorithmic}
\floatname{algorithm}{Algorithmus}
\caption{Algorithmus für zwei-orthogonale Achsen beim bewegende Ziele in TSP}
\label{alg:2OA.1}
\State \textbf{Input:} Ziele $T$, Ursprung $z_{ursprung}$, Verfolgergeschwindigkeit $v_{max}$
\State \textbf{Output:} Ziele $T$ in der Tour-Reihenfolge, inklusive Retour zum Ursprung\\

\State Sei \emph{t} das Zeit-Array, welches für jedes $z_i\in Z$ die Abfangzeit angibt
\State Sei \emph{current} das Ziel, welches der Verfolger soeben eingeholt hat
\State Sei \emph{OUTPUT} die Liste an Zielen in der Abfangreihenfolge
\State $current\rightarrow origin$
\State $OUTPUT.add(current)$
\State 

\For {$z_i\in Z$}
\State $t[z_i] \rightarrow \infty$
\State $\mathcal{Q}.add(z_i)$
\State Berechne $\alpha(z_i)$
\EndFor\\

\While {$\mathcal{Q}$ is not empty}
\If {jedes verbleibende Ziel liegt auf einer der vier Seiten des Schnittpunktes}
\For {$z_i\in \mathcal{Q}$}
\State $t[z_i] \rightarrow$ Zeit von der aktuellen Position bis zum Einholen von $z_i$
\EndFor
\State Sortiere $\mathcal{Q}$ in aufsteigender Reihenfolge nach $t[z_i]$
\State Berechne Rückkehr zum Ursprung ausgehend vom letzten Ziel aus $\mathcal{Q}$
\State $OUTPUT.addAll(\mathcal{Q})$
\State break
\EndIf \\

\State Berechne $\alpha(z_i),~\forall z_i\in\mathcal{Q}$
\State Update $\mathcal{Q}$
\State $prev\rightarrow current$
\State $current \rightarrow \mathcal{Q}.poll()$
\State $t[current] \rightarrow time[prev] + \pi[prev\rightarrow current]$
\State $OUTPUT.add(current)$
\State Update die Position von jedem $z_i\in\mathcal{Q}$
\State $EingeholteZiele \rightarrow$ Ziele zwischen $prev$ und $current$
\State Sortiere $EingeholteZiele$ in aufsteigender Reihenfolge nach $t[z_i]$
\State $OUTPUT.addAll(EingeholteZiele)$
\EndWhile\\

\State $OUTPUT.add(z_{ursprung})$

\end{algorithmic}
\end{algorithm}
\end{minipage}
\end{document}