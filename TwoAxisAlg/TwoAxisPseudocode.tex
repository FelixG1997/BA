\documentclass{scrartcl}


% input encoding
\usepackage[utf8]{inputenc}

% new german spelling
\usepackage[ngerman]{babel}

% choose font
\usepackage[T1]{fontenc}
\usepackage{lmodern}

% KOMA-Script options
\KOMAoptions{%
  parskip=full,%
  fontsize=12pt,%
  DIV=calc}

% page options
\usepackage{geometry}
\geometry{a4paper, portrait, left=2.5cm, right=2cm, top=2cm, bottom=2cm, includefoot}

% color and images
\usepackage{xcolor}
\usepackage{graphicx}

% draw with tikz
\usepackage{tikz}
\usetikzlibrary{positioning,automata}

% quotes
\usepackage[german=guillemets]{csquotes}

% better enumerations
\usepackage{enumitem}

% title spacing fix
%\usepackage{titlesec}
%\titlespacing*{\subsection}{0pt}{2\baselineskip}{-0.5\baselineskip}
\RedeclareSectionCommand[
  beforeskip=-2\baselineskip,
  afterskip=.5\baselineskip]{section}
\RedeclareSectionCommand[
  beforeskip=-1.5\baselineskip,
  afterskip=.1\baselineskip]{subsection}
\RedeclareSectionCommand[
  beforeskip=-1\baselineskip,
  afterskip=.05\baselineskip]{subsubsection}
\RedeclareSectionCommand[
  beforeskip=.5\baselineskip,
  afterskip=-1em]{paragraph}
\RedeclareSectionCommand[
  beforeskip=-.5\baselineskip,
  afterskip=-1em]{subparagraph}

% math
\usepackage{amsmath}
\usepackage{amssymb}
\usepackage{amsthm}

% custom theorems
\usepackage{thmtools}
\declaretheorem[style=definition]{aufgabe}

% set special behaviour for hyperlinks in pdfs
\usepackage[breaklinks=true]{hyperref}

% display program code
\usepackage{listings}

\lstset{ %
  backgroundcolor=\color{white!98!black},  
  basicstyle=\ttfamily\footnotesize,
  breakatwhitespace=true,
  breaklines=true,
  commentstyle=\color{green!40!black},
  frame=single,
  keywordstyle=\color{blue!70!black},
  language=ML,
  numbers=left,
  numbersep=8pt,
  numberstyle=\scriptsize\color{black!80}, 
  rulecolor=\color{black!70},
  stepnumber=1,
  stringstyle=\color{purple!80!black},
  tabsize=2,
  title=\lstname,
  language=ML,
  belowskip=-0.8\baselineskip,
  aboveskip=-0.8 \baselineskip,
}

% Pseudocode
\usepackage{algorithm}
\usepackage{algpseudocode}

% positioning
\usepackage{float}

% document information
\title{Moving-Target TSP in two-orthogonal-axes}
\subtitle{Pseudocode}
\author{Felix Greuling (666020)}
\date{05. Oktober 2019}

\begin{document}
\maketitle

% if, state, for, endif, endfor, else
% \Procedure{CalculateFunSums}{$T, w, visited, V$}

\begin{minipage}{1\linewidth}
\begin{algorithm}[H]
\caption{Exact Algorithm for two-orthogonal-axes Moving-Target TSP}
\begin{algorithmic}
\State \textbf{Input:} The initial positions and velocities of n targets, and the maximum pursuer speed
\State \textbf{Output:} A time-optimal tour intercepting all targets, and returning back to the origin
~ \\
\State \textbf{Preprocessing}
\State Partition the list of targets into the targets on the left side, the right side, \textcolor{red}{the top side and the bottom side} of the origin \\
Sort the targets on the left into list \emph{Left} in order of nonincreasing speeds \\
Sort the targets on the right into list \emph{Right} in order of nonincreasing speeds \\
\textcolor{red}{Sort the targets on the top into list \emph{Top} in order of nonincreasing speeds }\\
\textcolor{red}{Sort the targets on the bottom into list \emph{Bottom} in order of nonincreasing speeds}\\
\textcolor{red}{Delete targets in \emph{Left, Right, Top and Bottom} which are closer to the origin than faster targets in this list. Don't remove targets which move towards the other direction so they are crossing the origin.} \\
\textcolor{red}{Add an empty Target to \emph{Left, Right, Top and Bottom} to the end of each list}
\\
%\If {\emph{Left} or \emph{Right} is empty} 
\If {\textcolor{red}{3 of the 4 lists are empty}}
\State Calculate the time required to intercept all remaining targets; and 
\State Go to the postprocessing step
\EndIf

\end{algorithmic}
\end{algorithm}
\end{minipage}

\begin{minipage}{1\linewidth}
\begin{algorithm}[H]
\begin{algorithmic}

\State \textbf{Main Algorithm}
\State Let $A_0$ be the start state \\
Let $A_{final}$ be the final state \\
STATE is the sorted list of states in order of nondecreasing sum of the indices of each state's targets in lists \emph{Left, Right, \textcolor{red}{Top and Bottom.}} \textcolor{red}{Leave out the case with 4 empty targets} \\
Place $A_0$ first in the list STATE \\
Place $A_{final}$ last in the list STATE \\
$t(A) \leftarrow \infty$ for any state $A\neq A_0$  \\
$t(A_0)\leftarrow0 $\\
$current \leftarrow 0$
~\\
\While{current $\leq$ the number of states in \emph{STATE}}
\State $A=STATE[current]$
\If {there are no transitions into $A$}
\State Increment \emph{current} and jump back to the beginning of the while loop
\EndIf
\If {for state $A$, all remaining targets are on one side of the origin \textcolor{red}{($A$ consists of exactly 
\State three empty targets)}}
\State $t(\tau_{final})\leftarrow$ time required to intercept the remaining targets (and return to the 
\State origin)
\Else
\State Calculate the \textcolor{red}{four transitions $\tau_{left}, \tau_{right}, \tau_{top}, \tau_{bottom}$ from state $A$ using 
\State lists \emph{Left, Right, Top, Bottom}}
\If {$t(A) + t(\tau_{left}) < t(A_{left})$}
\State $t(A_{left}) \leftarrow t(A) + t(\tau_{left})$
\EndIf
\If {$t(A) + t(\tau_{right}) < t(A_{right})$}
\State $t(A_{right}) \leftarrow t(A) + t(\tau_{right})$
\EndIf
\textcolor{red}{
\If {$t(A) + t(\tau_{top}) < t(A_{top})$}
\State $t(A_{top}) \leftarrow t(A) + t(\tau_{top})$
\EndIf
\If {$t(A) + t(\tau_{bottom}) < t(A_{bottom})$}
\State $t(A_{bottom}) \leftarrow t(A) + t(\tau_{bottom})$
\EndIf 
}
\EndIf
\State Increment \emph{current}
\EndWhile
\State $OUTPUT \leftarrow$ the reverse list of states from $A_{final}$ back to $A_0$

\end{algorithmic}
\end{algorithm}
\end{minipage}

\begin{minipage}{1\linewidth}
\begin{algorithm}[H]
\begin{algorithmic}

\State \textbf{Postprocessing}
\For {pair of consecutive states in $OUTPUT$}
\State Calculate which targets are intercepted between the state pair 
\State Sort the intercepted targets by the interception order
\EndFor
\State Output the concatenated sorted lists of targets

\end{algorithmic}
\end{algorithm}
\end{minipage}

%\section*{Laufzeit}





\end{document}